%//////////////////////////////////////////////////////////////////////////////
%
% Copyright (c) 2007,2010 Daniel Adler <dadler@uni-goettingen.de>, 
%                         Tassilo Philipp <tphilipp@potion-studios.com>
%
% Permission to use, copy, modify, and distribute this software for any
% purpose with or without fee is hereby granted, provided that the above
% copyright notice and this permission notice appear in all copies.
%
% THE SOFTWARE IS PROVIDED "AS IS" AND THE AUTHOR DISCLAIMS ALL WARRANTIES
% WITH REGARD TO THIS SOFTWARE INCLUDING ALL IMPLIED WARRANTIES OF
% MERCHANTABILITY AND FITNESS. IN NO EVENT SHALL THE AUTHOR BE LIABLE FOR
% ANY SPECIAL, DIRECT, INDIRECT, OR CONSEQUENTIAL DAMAGES OR ANY DAMAGES
% WHATSOEVER RESULTING FROM LOSS OF USE, DATA OR PROFITS, WHETHER IN AN
% ACTION OF CONTRACT, NEGLIGENCE OR OTHER TORTIOUS ACTION, ARISING OUT OF
% OR IN CONNECTION WITH THE USE OR PERFORMANCE OF THIS SOFTWARE.
%
%//////////////////////////////////////////////////////////////////////////////

\newpage
\section{\emph{Dyncallback} C library API}

This library extends \product{dyncall} with function callback support, allowing
the user to dynamically create a callback object that can be called directly,
or passed to functions expecting a function-pointer as argument.
Once again, the flexibility is constrained by the set of supported types.

For style conventions and supported types, see \product{dyncall} API section.
In order to use \product{dyncallback}, include {\tt "dyncall\_callback.h"}.

\subsection{Callback Object}

The \emph{Callback Object} is the core component to this library.

\paragraph{Types}

\begin{lstlisting}[language=c]
typedef struct DCCallback DCCallback;
\end{lstlisting}

\paragraph{Details}
The \emph{Callback Object} is an object that mimics a fully typed function
call to another function (a generic callback handler, in this case). The
object itself can be created dynamically.

\subsection{Allocation}

\paragraph{Functions}

\begin{lstlisting}[language=c]
DCCallback* dcbNewCallback(const char* signature, DCCallbackHandler* funcptr, void* userdata);
void        dcbFreeCallback(DCCallback* pcb);
\end{lstlisting}

\lstinline{dcbNewCallback} creates and initializes a new \emph{Callback} object,
where \lstinline{signature} is the needed function signature (format is the
one outlined in the language bindings-section of this manual) of the function
to mimic, \lstinline{funcptr} is a pointer to a callback handler, and
\lstinline{userdata} a pointer to custom data that might be useful in the handler.
Use \lstinline{dcbFreeCallback} to destroy the \emph{Callback} object.

